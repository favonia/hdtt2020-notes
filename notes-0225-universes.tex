\documentclass[11pt]{article}
\usepackage{hdtt2020}
\usepackage{newpxmath}
\usepackage{newpxtext}

\newcommand{\U}{\ensuremath{\mathcal{U}}}

\newcommand{\wf}[1]{\ensuremath{\textnormal{well-founded}(#1)}}
\newcommand{\isequiv}[1]{\ensuremath{\textnormal{is-equiv}(#1)}}
\newcommand{\qinv}[1]{\ensuremath{\textnormal{qinv}(#1)}}
\newcommand{\isprop}[1]{\ensuremath{\textnormal{is-prop}(#1)}}
\newcommand{\iscontr}[1]{\ensuremath{\textnormal{is-contr}(#1)}}
\newcommand{\ishae}[1]{\ensuremath{\textnormal{is-hae}(#1)}}
\newcommand{\ap}{\textnormal{ap}}
\newcommand{\idtoequiv}{\textnormal{idtoequiv}}
\newcommand{\funapp}{\textnormal{happly}}

\title{CSCI 8980 Higher-Dimensional Type Theory\\ Lecture Notes}
\author{Dawn Michaelson, Jack Perisich}
\date{February 25, 2020}

\begin{document}

\maketitle





\section{Universe}
To internalize the $\istype{\Gamma}{A}$ judgment, we introduce the
universe type \U.  This can be viewed as the ``type of types''.  We
replace the original judgment with $\oftype{\Gamma}{A}{\U}$.
%
We can make this replacement in all our existing rules.  For example,
we originally had the following rule for dependent function types:
\begin{prooftree*}
  \hypo{\istype{\Gamma}{A}}
  \hypo{\istype{\Gamma, x:A}{B}}
  \infer2{\istype{\Gamma}{\dfuntype{x}{A}{B}}}
\end{prooftree*}
We can replace each judgment with a judgment that the given type has
the type of the universe, giving us a new rule:
\begin{prooftree*}
  \hypo{\oftype{\Gamma}{A}{\U}}
  \hypo{\oftype{\Gamma, x:A}{B}{\U}}
  \infer2{\oftype{\Gamma}{\dfuntype{x}{A}{B}}{\U}}
\end{prooftree*}
Doing this in all rules gives a system where we do not need the
original is-type judgment.





\section{Inconsistency}
Because \U{} is a type, we should have a rule:
\begin{prooftree*}
  \infer0{\oftype{\Gamma}{\U}{\U}}
\end{prooftree*}
With this rule, we can prove inconsistency, meaning a closed term of
the empty type.  This is the \textbf{Burali-Forti Paradox}.
\begin{enumerate}
\item Consider types with a transitive, well-founded order.  We will
  write this as \wf{A}.  This means that the type has a minimal
  element in any non-empty subset.
\item We define a type $W := \dpairtype{A}{\U}{\wf{A}}$.  This is a
  type which contains all well-founded types.
\item This type $W$ is well-founded with embedding as the relation.
  \begin{itemize}
  \item An embedding is an order-preserving function with a strict
    upper bound.  If we have types with ordering relations $(A, <_A)$
    and $(B, <_B)$ with an embedding $\oftype{f}{A \rightarrow B}$,
    $f$ will map ordered elements from $A$ to the same order in $B$,
    and there is some $b$ such that $f(a) <_B b$ for any $a$.
  \end{itemize}
  We can prove an embedding is transitive and well-founded.  Then we
  have $\oftype{(W, \textnormal{embedding})}{W}$.
\item Every well-founded type $(A, <_A)$ can be embedded into $W$.  We
  can define a map $a \mapsto \dpairtype{a'}{A}{a' <_A a}$.
\item Because of the previous point, $W$ can then be embedded into
  itself, yielding $W <_W W$.  This contradicts that embedding is
  well-founded.
\end{enumerate}
This was originally done in type theory by
Girard~\cite{girard_paradox_girard}.  It was later improved by
Coquand~\cite{girard_paradox_coquand} and
Hurkens~\cite{girard_paradox_hurkens}.





\section{Repairing Universes}
To remove this inconsistency, the universe \U{} becomes a hierarchy of
universes $\U_0, \U_1, \U_2, ...$.  We have two rules for these
universes:
\begin{mathpar}
  \begin{varwidth}{\textwidth}
    \begin{prooftree}
      \infer0{\oftype{\Gamma}{\U_i}{\U_{i+1}}}
    \end{prooftree}
  \end{varwidth}
  \and
  \begin{varwidth}{\textwidth}
    \begin{prooftree}
      \hypo{\oftype{\Gamma}{A}{\U_i}}
      \infer1{\oftype{\Gamma}{A}{\U_{i+1}}}
    \end{prooftree}
  \end{varwidth}
\end{mathpar}
A universe is contained in the next universe above it, and the
hierarchy is cumulative, so any type is contained in all the universes
above it.

How does this work when a rule has multiple premises judging that
types are members of some universe?  Above, before introducing a
universe hierarchy, we had a rule for dependent function types:
\begin{prooftree*}
  \hypo{\oftype{\Gamma}{A}{\U}}
  \hypo{\oftype{\Gamma, x:A}{B}{\U}}
  \infer2{\oftype{\Gamma}{\dfuntype{x}{A}{B}}{\U}}
\end{prooftree*}
There are two options for writing such a rule in the hierarchy of
universes.  The first is to take the least upper bound of the indices
of the premises ($i \sqcup j$) as the index of the conclusion:
\begin{prooftree*}
  \hypo{\oftype{\Gamma}{A}{\U_i}}
  \hypo{\oftype{\Gamma, x:A}{B}{\U_j}}
  \infer2{\oftype{\Gamma}{\dfuntype{x}{A}{B}}{\U_{i \sqcup j}}}
\end{prooftree*}
The second is to have the same index for both premises and the
conclusion:
\begin{prooftree*}
  \hypo{\oftype{\Gamma}{A}{\U_i}}
  \hypo{\oftype{\Gamma, x:A}{B}{\U_i}}
  \infer2{\oftype{\Gamma}{\dfuntype{x}{A}{B}}{\U_i}}
\end{prooftree*}
Because we have the rule for raising a type from one universe to the
next, we can raise one premise to match the other, then use this rule,
so neither rule is more powerful, and it is simply a stylistic
choice.

With this hierarchy and this principle for creating judgments of types
being in universes, we are unable to embed $W$ in itself, only in a
version of $W$ for a higher universe.
%
This is because the definition of the type $W$ requires the following
rule for placing it in a universe, once we have the universe
hierarchy:
\begin{prooftree*}
  \hypo{\oftype{\Gamma}{\U_i}{\U_j}}
  \hypo{\oftype{\Gamma, A:\U_i}{\wf{A}}{\U_j}}
  \infer2{\oftype{\Gamma}{\dpairtype{A}{\U_i}{\wf{A}}}{\U_j}}
\end{prooftree*}
Because $\U_i$, the type of $A$, is an element of $\U_j$, $i<j$, and
this dependent pair type cannot be an element of itself.
%
Because of this, we cannot recreate the Burali-Forti Paradox with
these rules.

In practice, proofs which do not distinguish between universes in the
hierarchy, acting as though the judgment $\oftype{\Gamma}{\U}{\U}$
holds, can usually have universe levels inserted consistently.  For
this reason, indices are often not shown when they do not affect the
proof.





\section{Univalence Principle}
The univalence principle is roughly that equivalent things should be
identified.  A good approximation of this is:
\[(A \simeq B) \simeq \idtype{\U}{A}{B}\]
This requires that we define equivalences $A \simeq B$.  We define
this as
\[(A \simeq B) := \dpairtype{f}{A \rightarrow B}{\isequiv{f}}\]
This, in turn, requires us to give a good definition of $\isequiv{}$.


\subsection{Function Equivalence}
A quasi-inverse of a function $\oftype{f}{A \rightarrow B}$
($\qinv{f}$) includes
\begin{itemize}
\item $\oftype{g}{B \rightarrow A}$
\item $\oftype{\epsilon}{\dfuntype{y}{B}{\idtype{B}{f(g(y))}{y}}}$
\item $\oftype{\eta}{\dfuntype{x}{A}{\idtype{A}{g(f(x))}{x}}}$
\end{itemize}
This may be formulated as a rather large dependent pair type.

For a good definition of $\isequiv{f}$, we want two things:
\begin{enumerate}
\item The definition should be logically equivalent to $\qinv{f}$
  (either one can be derived given the other)
\item $\isprop{\isequiv{f}}$ \\
  This requires that $\isequiv{f}$ has at most one element.  The
  definition of $\qinv{f}$ fails this requirement.
\end{enumerate}

We give two possible types to fulfill the requirements on
$\isequiv{f}$:
\begin{enumerate}
\item Half Adjoint Equivalence ($\ishae{f}$)

  This includes $g$, $\epsilon$, and $\eta$ as in the quasi-inverse,
  but it also includes a fourth element $\tau$:
  \[\oftype{\tau}{\dfuntype{x}{A}{\idtype{}{\ap_f(\eta(x))}{\epsilon(f(x))}}}\]
  This $\tau$ may be thought of as ensuring that $\eta$ and $\epsilon$
  are coherent in some way.

  It is clear how to derive $\qinv{f}$ given $\ishae{f}$, since we
  already have the $g$, $\epsilon$, and $\eta$.  The other direction
  of the equivalence is much more difficult to prove.

  Rather than $\tau$, we could have defined $\tau'$:
  \[\oftype{\tau'}{\dpairtype{y}{B}{\idtype{}{\eta(g(y))}{\ap_g(\epsilon(y))}}}\]
  This definition is dual to the definition of $\tau$.  It is the
  other ``half'' in the name of the type.
  %
  If the definition required both $\tau$ and $\tau'$, it would not
  satisfy the requirement of $\isprop{\isequiv{f}}$, and we would
  require a further element to show that $\tau$ and $\tau'$ were
  coherent.

\item $\isequiv{f} := \dfuntype{y}{B}{\iscontr{\dpairtype{x}{A}{\idtype{}{f(x)}{y}}}}$

  The inner dependent pair type is defining the pre-image of $y$ (in
  homotopy theory, this is the ``fiber over $y$'' or that ``all fibers
  are contractible'').  There is only one element in the pre-image,
  which gives an inverse function.
\end{enumerate}
Both of these definitions are acceptable.  In Agda, we prefer half
adjoint equivalence because it is easier to use.

A difficulty is that proving $\isprop{\isequiv{f}}$ for many
definitions requires function extensionality, which cannot be proved
without univalence.  We choose a definition of $\isequiv$ and use it
to define univalence, then prove $\isprop$ for it assuming univalence.


\subsection{Univalence and Function Extensionality}
A precise formulation of univalence:
\begin{tabbing}
  xxxxx\= \kill
  \> Define $\oftype{\idtoequiv}{\idtype{\U}{A}{B} \rightarrow A \simeq B}$ \\
  \> Axiom:  $\isequiv{\idtoequiv}$ \\
  \> $\idtoequiv := \lambda p.\idJ{A}{\textnormal{a proof of \isequiv{id_A}}}{p}$
\end{tabbing}
We define function extensionality as follows, using it to prove $\isprop{\isequiv{}}$.
\begin{tabbing}
  xxxxx\= \kill
  \> (Strong) Function Extensionality \\
  \> Define $\oftype{\funapp}{\idtype{f}{g}} \rightarrow \dfuntype{x}{A}{\idtype{f(x)}{g(x)}}$ \\
  \> Axiom:  $\isequiv{\funapp}$
\end{tabbing}

Some points to note:
\begin{enumerate}
\item By including these two axioms, we are breaking harmony.  This is
  fixed by cubical type theory.
\item Univalence can be viewed as an extensionality principle for
  universes.
\item Univalence implies function extensionality, so we don't need the
  axiom for $\funapp$.
\item Univalence implies $\lnot\textnormal{axiom k}$ and
  $\lnot\textnormal{LEM}$.  If we had the law of the excluded
  middle, that would imply that every type has decidable equality,
  which would in turn imply that every type is a set.  This is a
  contradiction, since \U{} is not a set.
\end{enumerate}


\printbibliography


\end{document}

